\section{Introduction}
\subsection{Overview}

Deep learning has revolutionised the field of computer vision in profound ways and has allowed many sectors that rely on image understanding to make significant progress. Computer vision tasks can be broken down into three main categories, image classification, object detection and localisation and semantic segmentaion. Semantic segmentation (also referred to as dense prediction) is a computer vision technique used to identify and label specific parts of an image at pixel level. The main goal is to precisely partition an image into areas (segments) that correspond to different objects or classes of interest (these could be for example person, car, tree, road, building etc.). This technique is widely used in various fields like for instance, in autonomous driving, where it helps cars understand the  environment around them, as well as in medical imaging, remote sensing, and robotics. 

Semantic segmentation is typically performed using deep learning methods, particularly convolutional neural networks (CNNs) and their variants as well as more recent attention-based models. In this study we have deployed the freely available Cityscapes dataset \cite{DBLP:journals/corr/CordtsORREBFRS16} to perform semantic segmentation analysis on urban scene imagery using a variety of Deep Learning architectures. Our goal is to train a model that will be able to receive an image of an urban landscape as input and produce as output an overlay mask that will identify the different types of objects with a high degree of accuracy. This task is of high importance, especially as many car manufacturers are trying to make their cars safer for their passengers as well as pedestrians and other road users. A system that uses this type of deep learning technology could be used for example to alert drivers when a pedestrian or a bicycle is detected in the projected path of the car. 
