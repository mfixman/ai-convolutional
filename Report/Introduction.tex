\section{Introduction}
\subsection{Overview}

In this study we have attempted to harness the power of deep learning in order to be able to analyse and understand the content of an image. More specifically, our aim was to create a program that, provided with an image of an urban environment, will be able to produce a precise identification of the different elements that make it up. In computer vision literature this task is called image segmentation. In image segmentation (often also referred to as dense prediction) the goal is to break down an image at pixel level into areas (segments) that correspond to different classes of objects. In our case the type of these classes would correspond to parts of the urban landscape, such as road, building, person, car, tree etc. A tool that allows to perform this task could have several applications. For example, it could be used to enhance vehicle security both for passengers and other road users by being able to promptly alert drivers when a pedestrian or bicycle is detected in the projected path of the car. It could also be very useful in creating applications for vehicle autonomy, where a car is able to identify the environment and trace a safe path ahead. 

Semantic segmentation is typically performed using deep learning methods, particularly convolutional neural networks (CNNs) and their variants as well as more recent attention-based models. In this study we have deployed the freely available Cityscapes dataset \cite{DBLP:journals/corr/CordtsORREBFRS16} which is a state-of-the-art dataset for this type of analysis. We have created and trained three different models, with distinct architectures and have evaluated their performance. We have performed a hyperparameter sweep analysis to find the set of values that gives us the best results. 

